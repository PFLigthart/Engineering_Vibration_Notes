\chapter{Chapter 4 - Multiple-degree-of-freedom systems}

More than one degree of freedom means more than one natural frequency. To keep record of each coordinate in the system, vectors are introduced and used along with matrices.

  \section{Two-degree-of-freedom model (undamped)}
    In moving from single-degree-of-freedom systems to two or more degrees of freedom, two important physical phenomena result.

    \begin{enumerate}
      \item The first important difference is that ta two-degree-of-freedom system will have two natural frequencies
      \item The second important phenomenon is that of a mode shape, which is not present in single-degree-of-freedom systems. A mode shape is a vector that described the relative motion between the two masses or between tow degrees of freedom.
    \end{enumerate}

  These important concepts of multiple natural frequencies and mode shapes are intimately tied to the mathematical concepts of eigenvalues and eigenvectors of computational matrix theory.


    \begin{figure}
      \centering
      \begin{tikzpicture}
        % single DOF mass-spring system
        \draw[thick] (0,0) -- (0,1);  % fixed wall
        % add hatching to represent the wall
        \fill[pattern=north east lines] (0,0) rectangle (-0.25,1);
        \draw[decorate,decoration=zigzag] (0,0.5) -- (2,0.5);
        \node at (1,-0.15) {$k_{1}$};  % spring constant label
        \draw (2,0) rectangle (3,1);
        \node at (2.5,0.5) {$m_{1}$};  % mass label
        \draw (2.5, 1.1) -- (2.5, 1.7);  % reference line 
        \draw[->, -{stealth}] (2.5, 1.4) -- (3.5, 1.4);  % coordinate arrow 
        \node at (3.75, 1.4) {$x_{1}$};  % coordinate label
        % draw a second mass connected to the first mass in series
        \draw[decorate,decoration=zigzag] (3,0.5) -- (5,0.5);
        \node at (4,-0.15) {$k_{2}$};  % spring constant label
        \draw (5,0) rectangle (6,1);
        \node at (5.5,0.5) {$m_{2}$};  % mass label
        \draw (5.5, 1.1) -- (5.5, 1.7);  % reference lines
        \draw[->, -{stealth}] (5.5, 1.4) -- (6.5, 1.4);  % coordinate arrow
        \node at (6.75, 1.4) {$x_{2}$};  % coordinate label
      \end{tikzpicture}
      \caption{Single degree of freedom mass-spring system}\label{fig:simple_2_dof_model_of_two_masses_connected_in_series_by_two_springs}
    \end{figure}

  \Cref{fig:simple_2_dof_model_of_two_masses_connected_in_series_by_two_springs} shows a simple two-degree-of-freedom system. The two masses are connected in series by two springs. Each mass only has a single degree of freedom because it can only move in a single direction. However, when considering the effect of the masses on one another, to completely described the system, two coordinates, namely $x_{1}$ and $x_{2}$ are needed. That is to say, the two coordinates depend on each other. The position of $m_2$ cannot be fully described without the position of $m_1$.

  \begin{figure}
    \centering
    \begin{tikzpicture}
      \draw[thick] (5,0) -- (5,1);  % fixed wall
      \fill[pattern=north east lines] (5,0) rectangle (5.25,1); % fixed wall hatching
      \draw (2,0) rectangle (3,1);  % mass
      \draw[decorate,decoration=zigzag] (3,0.5) -- (5,0.5);
      \draw[->, -{stealth}] (2, 0.5) -- (1, 0.5);  % horisontal coordinate arrow 
      \node at (4,-0.15) {$k_{2}$};  % spring constant label
      \node at (0.5, 0.5) {$x_{2}$};  % coordinate label
      \draw[decorate,decoration=zigzag] (2.5,1) -- (2.5,3);
      \draw[->, -{stealth}] (2.5, 0) -- (2.5, -1);  % vertical coordinate arrow
      \node at (2.5, -1.5) {$x_{1}$};  % coordinate label
      \draw[thick] (2,3) -- (3,3);  % fixed wall
      \fill[pattern=north east lines] (2,3) rectangle (3,3.25); % fixed wall hatching
      \node at (3,2) {$k_{1}$};  % spring constant label
    \end{tikzpicture}
    \caption{A single mass with two degrees of freedom, i.e. the mass moves along both the $x_1$ and $x_2$ directions}\label{fig:single_mass_with_two_degrees_of_freedom}
  \end{figure}

  \Cref{fig:single_mass_with_two_degrees_of_freedom} shows a single mass with two degrees of freedom. The mass can move in both the $x_1$ and $x_2$ directions. The mass is connected to a fixed wall by two springs.

  \begin{figure}
    \centering
    \begin{tikzpicture}
      \fill[pattern=north east lines] (2,3) rectangle (3,3.25); % fixed wall hatching
      \draw[decorate,decoration=zigzag] (2.5,1) -- (2.5,3);
      \draw (2,0) rectangle (3,1);  % mass
      \node at (2.5,0.5) {$m$};  % mass label
      \node at (3.5,2) {$k$, $k_{\theta}$};  % spring constant label
      \draw[thick] (2,3) -- (3,3);  % fixed wall
      \draw (3,0.5) -- (5.75,0.5); % rotational axis
      \draw[->, -{stealth}] (3.5, 0.5) -- (3.5, -1);  % vertical coordinate arrow
      \node at (3.5, -1.25) {$x$};  % coordinate label
      \draw[->, -{stealth}] (5, -0.35) arc[start angle=-90, end angle=-360, x radius=0.5, y radius=0.85];
      \node at (5.75, 1) {$\theta$};  % coordinate label 
    \end{tikzpicture}
    \caption{A single mass with one translation and one rotational degree of freedom.}\label{fig:single_mass_with_one_translation_and_one_rotational_degree_of_freedom}
  \end{figure}

  Each of the figures above, namely \cref{fig:simple_2_dof_model_of_two_masses_connected_in_series_by_two_springs,fig:single_mass_with_two_degrees_of_freedom,fig:single_mass_with_one_translation_and_one_rotational_degree_of_freedom} shows a two-degree-of-freedom system. Each of these systems requires more than one coordinate to describe the vibration.
