\chapter*{Chapter 2 - Response to harmonic excitation}
  \begin{fmd-definition}{Harmonic excitation}
    Harmonic excitation refers to a sinusoidal external force of a single frequency applied to the system.
  \end{fmd-definition}

  \begin{fmd-definition}{Resonance}
    Resonance is the tendency of a system to absorb more energy when the driving frequency matches the systems natural frequency of vibration.
  \end{fmd-definition}

  Harmonic excitations are a common source of external force applied to machines and structures. Rotating machines such as fans, electric motors, and reciprocating engines transmit a sinusoidally varying force to adjacent component's. In addition, the Fourier theorem indicates that many other forcing functions can be expressed as an infinite series of harmonic terms. Since the equations of motion considered here are linear, knowing the response to individual terms in the series allows the total repose to be represented as the sum of the response to the individual terms. This is the principle of superposition. In the way, knowing the response to a single harmonic input allows the calculation of the response to a variety of other input disturbances of periodic nature.

  For now, lets consider the driving force, $F(t)$ to be of the form:

  \begin{equation}
    F(t) = F_0 \cos(\omega t).
  \end{equation}

  \noindent Other forms could also have been chosen, but for now this will suffice.

  \begin{figure}
    \begin{tikzpicture}
      % draw a horisontal spring mass system with a force F_0 cos(wt) applied to the mass 
    \end{tikzpicture}
  \end{figure}
  
